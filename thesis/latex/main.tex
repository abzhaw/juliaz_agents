% =============================================================================
% Masterarbeit: Agentenbasierte KI-Systeme
% Multi-Agent-Orchestrierung am Beispiel von juliaz_agents
% Autor: Raphael
% =============================================================================

\documentclass[12pt, a4paper, twoside, openright]{report}

% --- Sprache & Encoding ---
\usepackage[utf8]{inputenc}
\usepackage[T1]{fontenc}
\usepackage[ngerman]{babel}
\usepackage[autostyle=true, german=quotes]{csquotes}

% --- Typografie ---
\usepackage{lmodern}
\usepackage{microtype}
\usepackage{setspace}
\onehalfspacing

% --- Seitenlayout ---
\usepackage[
  a4paper,
  inner=3cm,        % Bundsteg (innerer Rand)
  outer=2.5cm,
  top=2.5cm,
  bottom=2.5cm
]{geometry}

% --- Kopf- und Fußzeilen ---
\usepackage{fancyhdr}
\pagestyle{fancy}
\fancyhf{}
\fancyhead[LE]{\leftmark}
\fancyhead[RO]{\rightmark}
\fancyfoot[C]{\thepage}
\renewcommand{\headrulewidth}{0.4pt}

% --- Literaturverwaltung (BibLaTeX + Biber) ---
\usepackage[
  backend=biber,
  style=alphabetic,
  sorting=nyt,
  maxnames=3,
  minnames=1
]{biblatex}
\addbibresource{bibliography/references.bib}

% --- Code-Listings ---
\usepackage{listings}
\usepackage{xcolor}

\definecolor{codegreen}{rgb}{0,0.6,0}
\definecolor{codegray}{rgb}{0.5,0.5,0.5}
\definecolor{codepurple}{rgb}{0.58,0,0.82}
\definecolor{backcolour}{rgb}{0.97,0.97,0.97}

\lstdefinestyle{thesis}{
  backgroundcolor=\color{backcolour},
  commentstyle=\color{codegreen},
  keywordstyle=\color{codepurple}\bfseries,
  numberstyle=\tiny\color{codegray},
  stringstyle=\color{codegreen},
  basicstyle=\ttfamily\small,
  breakatwhitespace=false,
  breaklines=true,
  captionpos=b,
  keepspaces=true,
  numbers=left,
  numbersep=5pt,
  showspaces=false,
  showstringspaces=false,
  showtabs=false,
  tabsize=2,
  frame=single
}
\lstset{style=thesis}

% --- Abbildungen & Tabellen ---
\usepackage{graphicx}
\usepackage{float}
\usepackage{booktabs}
\usepackage{caption}
\usepackage{subcaption}
\graphicspath{{figures/}}

% --- TikZ für Diagramme ---
\usepackage{tikz}
\usetikzlibrary{arrows.meta, positioning, shapes.geometric, fit, calc}

% --- Hyperlinks ---
\usepackage[
  colorlinks=true,
  linkcolor=black,
  citecolor=blue!60!black,
  urlcolor=blue!60!black,
  pdfauthor={Raphael},
  pdftitle={Agentenbasierte KI-Systeme}
]{hyperref}
\usepackage[ngerman]{cleveref}

% --- Glossar (optional) ---
% \usepackage[acronym, toc]{glossaries}
% \makeglossaries

% =============================================================================
\begin{document}

% --- Titelblatt ---
\begin{titlepage}
  \centering
  \vspace*{2cm}

  {\Large Masterarbeit}\\[1.5cm]

  {\LARGE\bfseries Agentenbasierte KI-Systeme}\\[0.5cm]
  {\Large Multi-Agent-Orchestrierung\\am Beispiel von juliaz\_agents}\\[3cm]

  {\large vorgelegt von}\\[0.5cm]
  {\Large Raphael}\\[2cm]

  % TODO: Universität, Studiengang, Betreuer eintragen
  {\large
    Fachhochschule Vorarlberg\\
    Studiengang: [Studiengang]\\[1cm]
    Betreuer: [Name des Betreuers]\\
    Zweitprüfer: [Name des Zweitprüfers]\\[1cm]
    Dornbirn, \today
  }
\end{titlepage}

% --- Eidesstattliche Erklärung ---
\chapter*{Eidesstattliche Erklärung}
\thispagestyle{empty}

Ich erkläre hiermit an Eides statt, dass ich die vorliegende Masterarbeit selbstständig und ohne fremde Hilfe verfasst und keine anderen als die angegebenen Quellen und Hilfsmittel benutzt habe. Die aus fremden Quellen direkt oder indirekt übernommenen Gedanken sind als solche kenntlich gemacht.

Die Arbeit wurde bisher weder im In- noch im Ausland in gleicher oder ähnlicher Form einer anderen Prüfungsbehörde vorgelegt und auch noch nicht veröffentlicht.

\vspace{3cm}
\noindent
Dornbirn, am \hrulefill \\[1cm]
\noindent
\hrulefill \\
(Unterschrift)

\newpage

% --- Kurzfassung (Deutsch) ---
\chapter*{Kurzfassung}
\addcontentsline{toc}{chapter}{Kurzfassung}

% TODO: Deutsche Kurzfassung schreiben (max. 1 Seite)

\newpage

% --- Abstract (English) ---
\chapter*{Abstract}
\addcontentsline{toc}{chapter}{Abstract}

% TODO: English abstract (max. 1 page)

\newpage

% --- Inhaltsverzeichnis ---
\tableofcontents
\newpage

% --- Abbildungsverzeichnis ---
\listoffigures
\addcontentsline{toc}{chapter}{Abbildungsverzeichnis}
\newpage

% --- Tabellenverzeichnis ---
\listoftables
\addcontentsline{toc}{chapter}{Tabellenverzeichnis}
\newpage

% --- Listings-Verzeichnis ---
\lstlistoflistings
\addcontentsline{toc}{chapter}{Quellcodeverzeichnis}
\newpage

% =============================================================================
% HAUPTTEIL
% =============================================================================

% =============================================================================
% Kapitel 1: Einleitung
% =============================================================================

\chapter{Einleitung}
\label{ch:einleitung}

\section{Problemstellung und Motivation}
\label{sec:einleitung-motivation}

% TODO: Warum Multi-Agent-Systeme? Warum jetzt?
% - Aktuelle Entwicklungen in LLMs und Tool Use
% - Lücke zwischen einzelnen Chatbots und echten Agenten
% - Bedarf an praxisnaher Forschung

\section{Forschungsfragen}
\label{sec:einleitung-forschungsfragen}

% TODO: Forschungsfragen formulieren
% FF1: Wie lassen sich heterogene KI-Modelle in einem Multi-Agent-System orchestrieren?
% FF2: Welche Architekturmuster eignen sich für die Kommunikation zwischen autonomen Agenten?
% FF3: Wie kann Tool Use die Handlungsfähigkeit von KI-Agenten erweitern?

\section{Zielsetzung und Abgrenzung}
\label{sec:einleitung-ziel}

% TODO: Was wird untersucht, was explizit nicht?

\section{Aufbau der Arbeit}
\label{sec:einleitung-aufbau}

% TODO: Kapitelübersicht
% "Kapitel 2 legt die theoretischen Grundlagen..."

% =============================================================================
% Kapitel 2: Grundlagen
% =============================================================================

\chapter{Grundlagen}
\label{ch:grundlagen}

\section{Large Language Models}
\label{sec:grundlagen-llm}

% TODO: Transformer-Architektur, GPT, Claude — Überblick
% Fokus auf Eigenschaften, die für Agenten relevant sind

\section{Multi-Agent-Systeme}
\label{sec:grundlagen-mas}

% TODO: Definition, Geschichte, Taxonomie
% - Reaktive vs. deliberative Agenten
% - Kooperative vs. kompetitive Systeme
% - Kommunikationsprotokolle

\section{Tool Use und Function Calling}
\label{sec:grundlagen-tooluse}

% TODO: Konzept des Tool Use bei LLMs
% - OpenAI Function Calling
% - Anthropic Tool Use
% - ReAct-Pattern (Reasoning + Acting)

\section{Model Context Protocol (MCP)}
\label{sec:grundlagen-mcp}

% TODO: Anthropics MCP-Standard
% - Architektur: Client, Server, Transport
% - Tool-Definitionen, Resources, Prompts
% - Vergleich mit anderen Ansätzen

\section{Capability Routing}
\label{sec:grundlagen-routing}

% TODO: Konzept der modellbasierten Aufgabenverteilung
% - Welches Modell für welche Aufgabe?
% - Kosten-Leistungs-Abwägungen

% =============================================================================
% Kapitel 3: Verwandte Arbeiten
% =============================================================================

\chapter{Verwandte Arbeiten}
\label{ch:verwandte-arbeiten}

\section{Multi-Agent-Frameworks}
\label{sec:va-frameworks}

% TODO: Bestehende Frameworks analysieren
% - LangChain / LangGraph
% - AutoGen (Microsoft)
% - CrewAI
% - Semantic Kernel
% Jeweils: Architektur, Stärken, Schwächen, Vergleich mit juliaz_agents

\section{Agentic AI in der Forschung}
\label{sec:va-forschung}

% TODO: Akademische Arbeiten zu agentenbasierter KI
% - Voyager (NVIDIA)
% - Toolformer
% - ReAct
% - Chain-of-Thought und Reasoning

\section{Empathische KI-Systeme}
\label{sec:va-empathie}

% TODO: AI in Palliative Care, Dignity Therapy
% - SUPPORT-Studie
% - Wish Companion als Anwendungsfall

\section{Abgrenzung und Positionierung}
\label{sec:va-abgrenzung}

% TODO: Was unterscheidet juliaz_agents von bestehenden Ansätzen?
% - Praxisorientierung vs. Forschungsprototyp
% - Multi-Modell statt Single-Modell
% - Ambient Agents als neues Konzept

% =============================================================================
% Kapitel 4: Konzept und Architektur
% =============================================================================

\chapter{Konzept und Architektur}
\label{ch:konzept}

\section{Systemübersicht}
\label{sec:konzept-uebersicht}

% TODO: Gesamtarchitektur von juliaz_agents
% - 7 Komponenten und ihre Rollen
% - Abbildung: Systemarchitektur-Diagramm

\section{Julia — Der Orchestrator}
\label{sec:konzept-julia}

% TODO: Design-Entscheidungen
% - Warum GPT-4o als primäres Modell?
% - Tool-Calling-Architektur
% - Prompt-Design und Persönlichkeit

\section{Kommunikationsschicht}
\label{sec:konzept-kommunikation}

% TODO: Wie kommunizieren die Agenten?
% - Bridge (MCP-Relay)
% - Telegram-Gateway (OpenClaw)
% - Frontend (Next.js + Streaming)

\section{Multi-Modell-Orchestrierung}
\label{sec:konzept-multimodell}

% TODO: Capability Routing in der Praxis
% - Claude via MCP für komplexe Aufgaben
% - GPT-4o für Orchestrierung
% - Entscheidungslogik: welches Modell wann?

\section{Skill-basierte Agenten}
\label{sec:konzept-skills}

% TODO: SOUL.md-Pattern, SKILL.md-Dateien
% - Modularität durch Skills
% - Forschungsbasiertes Skill-Design
% - Beispiel: Wish Companion

\section{Ambient Agents}
\label{sec:konzept-ambient}

% TODO: ADHD Agent als proaktives System
% - Konzept: Agenten die ohne Aufforderung handeln
% - Human-in-the-Loop für destruktive Aktionen
% - Propose-Approve-Execute Pattern

% =============================================================================
% Kapitel 5: Implementierung
% =============================================================================

\chapter{Implementierung}
\label{ch:implementierung}

\section{Technologie-Stack}
\label{sec:impl-stack}

% TODO: Verwendete Technologien
% - TypeScript (Orchestrator, Bridge, Cowork-MCP)
% - Python (OpenClaw, ADHD Agent)
% - Next.js 15 (Frontend)
% - PostgreSQL + Docker (Backend)
% - PM2 (Prozessverwaltung)

\section{Orchestrator-Implementierung}
\label{sec:impl-orchestrator}

% TODO: Kernlogik in orchestrator/src/
% - openai.ts: generateReply() und Tool-Use-Loop
% - tools.ts: Tool-Definitionen
% - prompt.ts: Systemprompt-Generierung
% - index.ts: Server-Setup und Routing

\section{Tool-Calling-Mechanismus}
\label{sec:impl-toolcalling}

% TODO: Wie funktioniert Tool Use technisch?
% - Request → Tool Call → Result → Loop (max 5)
% - Beispiel: send_email
% - Fehlerbehandlung

\section{MCP-Integration}
\label{sec:impl-mcp}

% TODO: Bridge und Cowork-MCP
% - Stateless Transport (frischer Transport pro Request)
% - Claude als Sub-Agent via MCP
% - Implementierungsdetails

\section{Telegram-Gateway}
\label{sec:impl-telegram}

% TODO: OpenClaw-Implementierung
% - Python-basierter Bot
% - Skill-System
% - Verbindung zur Bridge

\section{Self-Modification}
\label{sec:impl-selfmod}

% TODO: /dev-Kommando
% - Git Pull → Install → Restart
% - Race Condition: Antwort vor Restart
% - Sicherheitsüberlegungen

\section{Entwicklungschronologie}
\label{sec:impl-chronologie}

% TODO: session-synthesizer nutzen
% - 21 Sessions in 2 Tagen
% - Meilensteine und Evolutionsschritte

% =============================================================================
% Kapitel 6: Evaluation
% =============================================================================

\chapter{Evaluation}
\label{ch:evaluation}

\section{Evaluationsmethodik}
\label{sec:eval-methodik}

% TODO: Wie wird das System evaluiert?
% - Qualitative Analyse (Fallstudien)
% - Funktionale Tests
% - Architekturanalyse

\section{Fallstudie: Wish Companion}
\label{sec:eval-wish}

% TODO: Evaluation der empathischen KI
% - Forschungsbasiertes Design vs. naive Implementierung
% - Ethische Überlegungen
% - Grenzen der KI in sensiblen Kontexten

\section{Fallstudie: ADHD Agent}
\label{sec:eval-adhd}

% TODO: Evaluation des Ambient Agent
% - Effizienz der Skill-Hygiene
% - Human-in-the-Loop Workflow
% - False Positives / Negatives

\section{Fallstudie: Multi-Modell-Routing}
\label{sec:eval-routing}

% TODO: Evaluation der Modellauswahl
% - Kosten-Leistungs-Analyse
% - Latenz-Vergleiche
% - Qualitätsunterschiede zwischen Modellen

\section{Diskussion}
\label{sec:eval-diskussion}

% TODO: Ergebnisse einordnen
% - Was funktioniert gut?
% - Was sind die Grenzen?
% - Generalisierbarkeit (N=1 Problem)

\section{Limitationen}
\label{sec:eval-limitationen}

% TODO: Ehrliche Einschränkungen
% - Ein System, ein Entwickler
% - Keine quantitative Nutzerstudie
% - Abhängigkeit von API-Anbietern

% =============================================================================
% Kapitel 7: Zusammenfassung und Ausblick
% =============================================================================

\chapter{Zusammenfassung und Ausblick}
\label{ch:zusammenfassung}

\section{Zusammenfassung}
\label{sec:zf-zusammenfassung}

% TODO: Kernergebnisse zusammenfassen
% - Forschungsfragen beantworten
% - Beitrag zur Forschung

\section{Beantwortung der Forschungsfragen}
\label{sec:zf-forschungsfragen}

% TODO: Jede Forschungsfrage explizit beantworten
% FF1: Multi-Modell-Orchestrierung → Capability Routing
% FF2: Kommunikation → MCP + Bridge-Pattern
% FF3: Tool Use → send_email, /dev, Ambient Agents

\section{Ausblick}
\label{sec:zf-ausblick}

% TODO: Zukünftige Arbeiten
% - Skalierung auf mehr Agenten
% - Quantitative Evaluation
% - Weitere Anwendungsdomänen
% - Ethik-Framework für empathische KI

\section{Reflexion}
\label{sec:zf-reflexion}

% TODO: Persönliche Reflexion
% - Was wurde gelernt?
% - Was würde man anders machen?
% - Meta-Aspekt: Die Thesis wurde mit Hilfe eines Agenten geschrieben,
%   der Teil des dokumentierten Systems ist


% =============================================================================
% ANHANG & LITERATUR
% =============================================================================

% --- Literaturverzeichnis ---
\printbibliography[heading=bibintoc, title={Literaturverzeichnis}]

% --- Anhang (optional) ---
% \appendix
% \input{chapters/anhang}

\end{document}
