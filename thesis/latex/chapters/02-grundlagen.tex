% =============================================================================
% Kapitel 2: Grundlagen
% =============================================================================

\chapter{Grundlagen}
\label{ch:grundlagen}

\section{Large Language Models}
\label{sec:grundlagen-llm}

% TODO: Transformer-Architektur, GPT, Claude — Überblick
% Fokus auf Eigenschaften, die für Agenten relevant sind

\section{Multi-Agent-Systeme}
\label{sec:grundlagen-mas}

% TODO: Definition, Geschichte, Taxonomie
% - Reaktive vs. deliberative Agenten
% - Kooperative vs. kompetitive Systeme
% - Kommunikationsprotokolle

\section{Tool Use und Function Calling}
\label{sec:grundlagen-tooluse}

% TODO: Konzept des Tool Use bei LLMs
% - OpenAI Function Calling
% - Anthropic Tool Use
% - ReAct-Pattern (Reasoning + Acting)

\section{Model Context Protocol (MCP)}
\label{sec:grundlagen-mcp}

% TODO: Anthropics MCP-Standard
% - Architektur: Client, Server, Transport
% - Tool-Definitionen, Resources, Prompts
% - Vergleich mit anderen Ansätzen

\section{Capability Routing}
\label{sec:grundlagen-routing}

% TODO: Konzept der modellbasierten Aufgabenverteilung
% - Welches Modell für welche Aufgabe?
% - Kosten-Leistungs-Abwägungen
