% =============================================================================
% Kapitel 4: Konzept und Architektur
% =============================================================================

\chapter{Konzept und Architektur}
\label{ch:konzept}

\section{Systemübersicht}
\label{sec:konzept-uebersicht}

% TODO: Gesamtarchitektur von juliaz_agents
% - 7 Komponenten und ihre Rollen
% - Abbildung: Systemarchitektur-Diagramm

\section{Julia — Der Orchestrator}
\label{sec:konzept-julia}

% TODO: Design-Entscheidungen
% - Warum GPT-4o als primäres Modell?
% - Tool-Calling-Architektur
% - Prompt-Design und Persönlichkeit

\section{Kommunikationsschicht}
\label{sec:konzept-kommunikation}

% TODO: Wie kommunizieren die Agenten?
% - Bridge (MCP-Relay)
% - Telegram-Gateway (OpenClaw)
% - Frontend (Next.js + Streaming)

\section{Multi-Modell-Orchestrierung}
\label{sec:konzept-multimodell}

% TODO: Capability Routing in der Praxis
% - Claude via MCP für komplexe Aufgaben
% - GPT-4o für Orchestrierung
% - Entscheidungslogik: welches Modell wann?

\section{Skill-basierte Agenten}
\label{sec:konzept-skills}

% TODO: SOUL.md-Pattern, SKILL.md-Dateien
% - Modularität durch Skills
% - Forschungsbasiertes Skill-Design
% - Beispiel: Wish Companion

\section{Ambient Agents}
\label{sec:konzept-ambient}

% TODO: ADHD Agent als proaktives System
% - Konzept: Agenten die ohne Aufforderung handeln
% - Human-in-the-Loop für destruktive Aktionen
% - Propose-Approve-Execute Pattern
