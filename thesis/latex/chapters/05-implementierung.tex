% =============================================================================
% Kapitel 5: Implementierung
% =============================================================================

\chapter{Implementierung}
\label{ch:implementierung}

\section{Technologie-Stack}
\label{sec:impl-stack}

% TODO: Verwendete Technologien
% - TypeScript (Orchestrator, Bridge, Cowork-MCP)
% - Python (OpenClaw, ADHD Agent)
% - Next.js 15 (Frontend)
% - PostgreSQL + Docker (Backend)
% - PM2 (Prozessverwaltung)

\section{Orchestrator-Implementierung}
\label{sec:impl-orchestrator}

% TODO: Kernlogik in orchestrator/src/
% - openai.ts: generateReply() und Tool-Use-Loop
% - tools.ts: Tool-Definitionen
% - prompt.ts: Systemprompt-Generierung
% - index.ts: Server-Setup und Routing

\section{Tool-Calling-Mechanismus}
\label{sec:impl-toolcalling}

% TODO: Wie funktioniert Tool Use technisch?
% - Request → Tool Call → Result → Loop (max 5)
% - Beispiel: send_email
% - Fehlerbehandlung

\section{MCP-Integration}
\label{sec:impl-mcp}

% TODO: Bridge und Cowork-MCP
% - Stateless Transport (frischer Transport pro Request)
% - Claude als Sub-Agent via MCP
% - Implementierungsdetails

\section{Telegram-Gateway}
\label{sec:impl-telegram}

% TODO: OpenClaw-Implementierung
% - Python-basierter Bot
% - Skill-System
% - Verbindung zur Bridge

\section{Self-Modification}
\label{sec:impl-selfmod}

% TODO: /dev-Kommando
% - Git Pull → Install → Restart
% - Race Condition: Antwort vor Restart
% - Sicherheitsüberlegungen

\section{Entwicklungschronologie}
\label{sec:impl-chronologie}

% TODO: session-synthesizer nutzen
% - 21 Sessions in 2 Tagen
% - Meilensteine und Evolutionsschritte
