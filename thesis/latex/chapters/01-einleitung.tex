% =============================================================================
% Kapitel 1: Einleitung
% =============================================================================

\chapter{Einleitung}
\label{ch:einleitung}

\section{Problemstellung und Motivation}
\label{sec:einleitung-motivation}

% TODO: Warum Multi-Agent-Systeme? Warum jetzt?
% - Aktuelle Entwicklungen in LLMs und Tool Use
% - Lücke zwischen einzelnen Chatbots und echten Agenten
% - Bedarf an praxisnaher Forschung

\section{Forschungsfragen}
\label{sec:einleitung-forschungsfragen}

% TODO: Forschungsfragen formulieren
% FF1: Wie lassen sich heterogene KI-Modelle in einem Multi-Agent-System orchestrieren?
% FF2: Welche Architekturmuster eignen sich für die Kommunikation zwischen autonomen Agenten?
% FF3: Wie kann Tool Use die Handlungsfähigkeit von KI-Agenten erweitern?

\section{Zielsetzung und Abgrenzung}
\label{sec:einleitung-ziel}

% TODO: Was wird untersucht, was explizit nicht?

\section{Aufbau der Arbeit}
\label{sec:einleitung-aufbau}

% TODO: Kapitelübersicht
% "Kapitel 2 legt die theoretischen Grundlagen..."
