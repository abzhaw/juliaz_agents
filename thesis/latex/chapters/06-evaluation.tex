% =============================================================================
% Kapitel 6: Evaluation
% =============================================================================

\chapter{Evaluation}
\label{ch:evaluation}

\section{Evaluationsmethodik}
\label{sec:eval-methodik}

% TODO: Wie wird das System evaluiert?
% - Qualitative Analyse (Fallstudien)
% - Funktionale Tests
% - Architekturanalyse

\section{Fallstudie: Wish Companion}
\label{sec:eval-wish}

% TODO: Evaluation der empathischen KI
% - Forschungsbasiertes Design vs. naive Implementierung
% - Ethische Überlegungen
% - Grenzen der KI in sensiblen Kontexten

\section{Fallstudie: ADHD Agent}
\label{sec:eval-adhd}

% TODO: Evaluation des Ambient Agent
% - Effizienz der Skill-Hygiene
% - Human-in-the-Loop Workflow
% - False Positives / Negatives

\section{Fallstudie: Multi-Modell-Routing}
\label{sec:eval-routing}

% TODO: Evaluation der Modellauswahl
% - Kosten-Leistungs-Analyse
% - Latenz-Vergleiche
% - Qualitätsunterschiede zwischen Modellen

\section{Diskussion}
\label{sec:eval-diskussion}

% TODO: Ergebnisse einordnen
% - Was funktioniert gut?
% - Was sind die Grenzen?
% - Generalisierbarkeit (N=1 Problem)

\section{Limitationen}
\label{sec:eval-limitationen}

% TODO: Ehrliche Einschränkungen
% - Ein System, ein Entwickler
% - Keine quantitative Nutzerstudie
% - Abhängigkeit von API-Anbietern
